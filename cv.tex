%%%%%%%%%%%%%%%%%%%%%%%%%%%%%%%%%%%%%%%%%
% Medium Length Professional CV
% LaTeX Template
% Version 2.0 (8/5/13)
%
% This template has been downloaded from:
% http://www.LaTeXTemplates.com
%
% Original author:
% Trey Hunner (http://www.treyhunner.com/)
%
% Important note:
% This template requires the resume.cls file to be in the same directory as the
% .tex file. The resume.cls file provides the resume style used for structuring the
% document.
%
%%%%%%%%%%%%%%%%%%%%%%%%%%%%%%%%%%%%%%%%%

%----------------------------------------------------------------------------------------
%	PACKAGES AND OTHER DOCUMENT CONFIGURATIONS
%----------------------------------------------------------------------------------------

% usenames and dvipsnames gives access to 68 dvips colors
% see https://en.wikibooks.org/wiki/LaTeX/Colors
\documentclass[usenames, dvipsnames]{resume} % Use the custom resume.cls style

\usepackage[left=0.75in,top=0.6in,right=0.75in,bottom=0.6in]{geometry} % Document margins
\usepackage{bibentry}

% testing multibib
\usepackage{multibib}

% make the headers for multibibs be subheading sized instead
\usepackage{etoolbox}
\BeforeBeginEnvironment{thebibliography}{%
  \let\origsection\section% save the original definition of \section
  \let\section\subsection%  make \section behave like \subsection
}
\AfterEndEnvironment{thebibliography}{%
  \let\section\origsection% restore the original definition of \section
}
\newcites{papers}{Papers}
\newcites{reports}{Technical Reports}
\newcites{patents}{Patents}

\name{Trevor Ablett} % Your name
\address{University of Toronto Institute for Aerospace Studies \\ 4925 Dufferin St. \\ Toronto, ON M3H 5T6} % Your address
%\address{123 Pleasant Lane \\ City, State 12345} % Your secondary addess (optional)
\address{(647)~$\cdot$~997~$\cdot$~8738 \\ trevor.ablett@robotics.utias.utoronto.ca \\ \url{trevorablett.github.io/research}} % Your phone number and email

\begin{document}
	
% \vspace{-6cm}

% testing multibib
\nocitepapers{
    2024_Ablett_vpace,
    2024_Ablett_Push,
    limoyoWorkingBackwardsLearning2023,
    2023_Ablett_Learning,
    2022_Limoyo_Learning,
    2021_Ablett_Learning,
    ablettSeeingAllAngles2021,
    2019_Maric_Fast,
    limoyo2018automatic}

\nocitereports{
    ablettFightingFailuresFIRE2020}

\nocitepatents{
    calibration_patent,
    callisto_patent}

% \bibliographystyle{IEEEcaps}
% \nobibliography{refs} % NOTE: to get this to work, when you modify the bib, switch this to \bibliography{refs} then back to \nobibliography{refs}
%\bibliography{refs}

%----------------------------------------------------------------------------------------
%	EDUCATION SECTION
%----------------------------------------------------------------------------------------

\begin{rSection}{Education}

\textbf{Ph.D} (in progress), University of Toronto, Toronto, Ontario \hfill {2017 - Present} \\
Institute for Aerospace Studies, Space and Terrestrial Autonomous Robotics Systems Lab \smallskip \\
\textit{Topic:} Addressing the Limitations of Imitation Learning in Robotic Manipulators \\
\textit{Supervisor:} Dr. Jonathan Kelly. \smallskip \\
Overall GPA: 4.0/4.0

\textbf{M.A.Sc.} (Transferred to PhD), University of Toronto, Toronto, Ontario \hfill {2016 - 2017} \\
Institute for Aerospace Studies, Space and Terrestrial Autonomous Robotics Systems Lab \smallskip \\
\textit{Topic:} Active Calibration of a Mobile Manipulator \\
\textit{Supervisor:} Dr. Jonathan Kelly. \smallskip \\
Overall GPA: 4.0/4.0

\textbf{B.Eng., Mechatronics}, McMaster University, Hamilton, Ontario \hfill {2011 - 2015} \\
Faculty of Engineering, Dept. of Computing and Science \smallskip \\
Summa cum laude, Overall GPA: 3.9/4.0

\textbf{B.A., Psychology}, McMaster University, Hamilton, Ontario \hfill {2009 - 2015} \\
Faculty of Social Sciences, Dept. of Psychology, Neuroscience and Behaviour \smallskip \\
Summa cum laude, Overall GPA: 3.9/4.0

\end{rSection}

%----------------------------------------------------------------------------------------
%	PUBLICATIONS SECTION
%----------------------------------------------------------------------------------------

\begin{rSection}{Publications}

\bibliographystylepapers{IEEEcaps}
\bibliographypapers{refs}

\bibliographystylereports{IEEEcaps}
\bibliographyreports{refs}

\bibliographystylepatents{IEEEcaps}
\bibliographypatents{refs}

\end{rSection}

% \begin{rEnumSection}{Publications}

% \item \bibentry{2024_Ablett_vpace}
% \item \bibentry{2024_Ablett_Push}
% \item \bibentry{limoyoWorkingBackwardsLearning2023}
% \item \bibentry{2023_Ablett_Learning}
% \item \bibentry{2022_Limoyo_Learning}
% \item \bibentry{2021_Ablett_Learning}
% \item \bibentry{ablettSeeingAllAngles2021}
% \item \bibentry{2019_Maric_Fast}
% \item \bibentry{limoyo2018automatic}
    
% \end{rEnumSection}

% \begin{rSection}{Publications}
	
% 	\begin{enumerate}
%             \item \bibentry{2024_Ablett_vpace}
% 		\item \bibentry{2024_Ablett_Push}
% 		\item \bibentry{limoyoWorkingBackwardsLearning2023}
% 		\item \bibentry{2023_Ablett_Learning}
% 		\item \bibentry{2022_Limoyo_Learning}
% 		\item \bibentry{2021_Ablett_Learning}
% 		\item \bibentry{ablettSeeingAllAngles2021}
% 		\item \bibentry{2019_Maric_Fast}
% 		\item \bibentry{limoyo2018automatic}
% 	\end{enumerate}
	
	
% 	%------------------------------------------------
	
% \end{rSection}

%----------------------------------------------------------------------------------------
%	TECHNICAL REPORTS SECTION
%----------------------------------------------------------------------------------------

% \begin{rEnumSection}{Technical Reports}

% \item \bibentry{ablettFightingFailuresFIRE2020}
    
% \end{rEnumSection}


% \begin{rSection}{Technical Reports}
	
% 	\begin{enumerate}
% 		\item \bibentry{ablettFightingFailuresFIRE2020}
% 	\end{enumerate}
	
% 	%------------------------------------------------
	
% \end{rSection}

%----------------------------------------------------------------------------------------
%	PATENTS SECTION
%----------------------------------------------------------------------------------------

% \begin{rEnumSection}{Patents}

% \item \bibentry{calibration_patent}
% \item \bibentry{callisto_patent}
    
% \end{rEnumSection}

% \begin{rSection}{Patents}
	
% 	\begin{enumerate}
% 		\item \bibentry{calibration_patent}
% 		\item \bibentry{callisto_patent}
% 	\end{enumerate}
	
	
% 	%------------------------------------------------
	
% \end{rSection}

%----------------------------------------------------------------------------------------
%	AWARDS SECTION
%----------------------------------------------------------------------------------------

\begin{rSection}{Awards}
	\begin{rSubsection}{Queen Elizabeth II Graduate Scholarship in Science and \\ Technology (QEII-GSST)}{September 2020 - August 2021}{University of Toronto}{Toronto, ON}
		\item \$5000 per semester, \$15000 total.
	\end{rSubsection}
	
	\begin{rSubsection}{Ontario Graduate Scholarship (OGS)}{September 2019 - August 2020}{University of Toronto}{Toronto, ON}
		\item \$5000 per semester, \$15000 total.
	\end{rSubsection}
	
	\begin{rSubsection}{Kenneth M. Molson Fellowship}{October 2019}{University of Toronto}{Toronto, ON}
		\item \$2500.
	\end{rSubsection}
	
	\begin{rSubsection}{Ontario Graduate Scholarship (OGS)}{September 2018 - August 2019}{University of Toronto}{Toronto, ON}
		\item \$5000 per semester, \$15000 total.
	\end{rSubsection}
	
	\begin{rSubsection}{Douglas Patton Hogg Memorial Award}{December 2018}{University of Toronto}{Toronto, ON}
		\item \$2531.
	\end{rSubsection}
	
	\begin{rSubsection}{Ontario Graduate Scholarship (OGS)}{September 2017 - April 2018}{University of Toronto}{Toronto, ON}
		\item \$5000 per semester, \$10000 total.
	\end{rSubsection}
	
	\begin{rSubsection}{University (Senate) Scholarship}{September 2013 - August 2014}{McMaster University}{Hamilton, ON}
		\item \$800.
	\end{rSubsection}
	
	\begin{rSubsection}{McMaster Honour Award, Level 3}{September 2009 - August 2011}{McMaster University}{Hamilton, ON}
		\item \$2000 per year, \$4000 total.
	\end{rSubsection}
	
	
	%------------------------------------------------
	
\end{rSection}

%----------------------------------------------------------------------------------------
%	TEACHING EXPERIENCE SECTION
%----------------------------------------------------------------------------------------

\begin{rSection}{Teaching Experience}

\begin{rTeachSubsection}{University of Toronto}{Winter 2018-Spring 2022}{Teaching Assistant}{Toronto, ON}{AER521 - Mobile Robotics}
	\item Robotics course with both undergraduate and graduate level students
	\item Developed, administered, and graded MATLAB/ROS robotics laboratories
\end{rTeachSubsection}

\begin{rTeachSubsection}{Coursera.org and University of Toronto}{October 2018 - April 2019}{Subject Matter Expert}{Toronto, ON}{Self-Driving Car Specialization}
	\item Developing code, assignments and other supplementary material for a course on state estimation of self-driving cars.
	\item Assignments are on sensor fusion using filtering techniques, point cloud matching, and 3D geometry.
\end{rTeachSubsection}


\begin{rTeachSubsection}{University of Toronto}{Winter 2018}{Teaching Assistant}{Toronto, ON}{APS106 - Fundamentals of Computer Programming}
	\item First year programming course using Python
	\item Administered weekly programming laboratories to students and aided in ongoing development of course
\end{rTeachSubsection}
	
	
\begin{rTeachSubsection}{University of Toronto}{Fall 2016}{Teaching Assistant}{Toronto, ON}{ROB501 - Computer Vision for Robotics}
	\item Course with both undergraduate and graduate level students
	\item Administered MATLAB and computer vision tutorials
	\item Aided in development and marking of MATLAB based computer vision assignments
\end{rTeachSubsection}

%------------------------------------------------

\begin{rTeachSubsection}{McMaster University}{Winter 2015}{Teaching Assistant}{Hamilton, ON}{Software Engineering 2DA4 - Digital Systems and Interfacing}
	\item Administered labs using Verilog HDL and Altera based FPGAs
\end{rTeachSubsection}

%------------------------------------------------

\begin{rTeachSubsection}{McMaster University}{Fall 2014}{Teaching Assistant}{Hamilton, ON}{Software Engineering 3I03 - Communications Skills}
	\item Created presentation materials for tutorials on giving software engineering presentations
	\item Ran weekly mandatory tutorials for 30 students
\end{rTeachSubsection}
	
\end{rSection}

%----------------------------------------------------------------------------------------
%	WORK EXPERIENCE SECTION
%----------------------------------------------------------------------------------------

\begin{rSection}{Work Experience}

\begin{rSubsection}{Samsung Research America}{September 2022 - November 2023}{Research Intern -- Applied Reinforcement Learning}{Montreal, QC, Canada}
	\item Development and implementation of new deep reinforcement and imitation learning algorithms to learn control policies to solve real-world robotics problems
	\item Development, improvement, and maintenance of software libraries for various learning, robotic, and sensing applications
\end{rSubsection}

\begin{rSubsection}{Callisto Mechanical}{April 2015 - June 2016}{Controls Engineer in Training}{Niagara-on-the-Lake, ON}
\item Management and execution of research based projects in vision, robotics, and controls
\item Named on pending patent for a vision-based Automated Guided Vehicle
\item Development of software based controls, HMIs, and SCADA for OEM machines to be used in process automation
\item Worked with various software and hardware tools, including Java and C++ based embedded systems, PLCs, and HMIs
\item Attended numerous sites for commissioning of various machines and software systems
\end{rSubsection}

\begin{rSubsection}{Self Employed -- University Level Private Tutor}{September 2013 - April 2015}{Introductory Level Programming}{Hamilton, ON}
	\item Charged a small fee for private tutoring sessions in an introductory level programming class where assignments were completed using Python.
\end{rSubsection}

\begin{rSubsection}{Callisto Integration}{May 2014 - August 2014}{Controls Engineer in Training}{Hamilton, ON}
	\item Lead designer of HMI for a Solar Farm
	\item PLC programming and debugging of existing systems
\end{rSubsection}

\begin{rSubsection}{Venture Engineering and Science Camp}{May 2013 - April 2014}{Computer/Technology/Robotics Instructor}{Hamilton, ON}
	\item Designed various electronics, computer, and robotics projects for elementary school aged children
\end{rSubsection}

%------------------------------------------------

\end{rSection}

%----------------------------------------------------------------------------------------
%	VOLUNTEER EXPERIENCE SECTION
%----------------------------------------------------------------------------------------

\begin{rSection}{Volunteer Experience}
	
	\begin{rSubsection}{Bay Area Science and Engineering Fair (BASEF)}{January 2017 - April 2017}{Team Mentor}{Burlington, ON}
		\item Provided weekly assistance and advice to an elementary school science fair team
	\end{rSubsection}
	
	%------------------------------------------------
	
	\begin{rSubsection}{Industry Education Council of Hamilton}{January 2015 - June 2015}{Code Club - Instructor}{Hamilton, ON}
		\item Ran a lunchtime club for elementary school students to learn programming through simple projects
	\end{rSubsection}
	
	%------------------------------------------------
	
\end{rSection}

%----------------------------------------------------------------------------------------
%	MEDIA APPEARANCES SECTION
%----------------------------------------------------------------------------------------

\begin{rSection}{Media Appearances}
	
	\begin{rSubsection}{Ridgeback Helping to Solve Challenging Mobile Manipulation Tasks}{Nov 18, 2020}{Clearpath Robotics}
		\item Clearpath Robotics wrote a blog post showcasing our lab and our mobile manipulation platform, including a video generated as part of a project of mine in which I used end-to-end policies to complete difficult tasks regardless of viewpoint.
		\href{https://clearpathrobotics.com/blog/2020/11/ridgeback-helping-to-solve-challenging-mobile-manipulation-tasks/}{[Blog post]}   \href{https://s3.amazonaws.com/assets.clearpathrobotics.com/wp-content/uploads/2020/11/17132718/thing_insertion_sped-up-4x-1.mp4?_=1}{[Video only]}
	\end{rSubsection}

	\begin{rSubsection}{Ontario Centres of Excellence (OCE) Showcase -- Demo }{Aired May 17, 2017}{China Central Television}
		\item CCTV-13, the Chinese national news channel, included a short segment in their daily broadcast with video of me teleoperating our mobile manipulator platform. 
		\href{http://tv.cctv.com/2017/05/17/VIDEVnRAiZppeYYkkeEqDz1u170517.shtml}{[Online news brief (Chinese)]}
	\end{rSubsection}
	
	%------------------------------------------------

\end{rSection}

%----------------------------------------------------------------------------------------
%	TECHNICAL STRENGTHS SECTION
%----------------------------------------------------------------------------------------

\begin{rSection}{Technical Strengths}

\begin{tabular}{ @{} >{\bfseries}l @{\hspace{6ex}} l }
Programming Languages & Python, C++, C, Java, MATLAB, LaTeX, Verilog, Ladder Logic \\
Frameworks/Libraries & numpy, scipy, tensorflow, pytorch, ROS, OpenCV, scikit-learn \\
Hardware & Arduino, Raspberry Pi, PIC microcontroller, various actuators and sensors \\
Tools & Linux (CLI), Windows, MS Office, Git, SVN \\ 
\end{tabular}

\end{rSection}

%----------------------------------------------------------------------------------------
%	EXAMPLE SECTION
%----------------------------------------------------------------------------------------

%\begin{rSection}{Section Name}

%Section content\ldots

%\end{rSection}

%----------------------------------------------------------------------------------------

\end{document}
